\documentclass[12pt]{standalone}
% Based on http://www.texample.net/tikz/examples/neural-network/

\usepackage{tikz}

\begin{document}

\def\layersep{2.5cm}

\begin{tikzpicture}[shorten >=1pt,->,draw=black!80, node distance=\layersep]
    \tikzstyle{every pin edge}=[<-,shorten <=1pt]
    \tikzstyle{neuron}=[circle,fill=black!25,minimum size=17pt,inner sep=0pt]
    \tikzstyle{annot} = [text width=4em, text centered]

    % Draw the input layer nodes
    \foreach \name / \y in {1/1,2/2,$d$/4}
    % This is the same as writing \foreach \name / \y in {1/1,2/2,3/3,4/4}
        \node[neuron] (I-\name) at (0,-\y) {};
    \node (I-3) at (0,-3) {\vdots};

    % Draw the hidden layer nodes
    \foreach \name / \y in {1/1,2/2,$K$/4}
        \node[neuron] (H-\name) at (\layersep,-\y cm) {};
    \node (H-3) at (\layersep,-3) {\vdots};

    \foreach \name / \y in {1/1,2/2,$M$/4}
        \node[neuron] (O-\name) at (2*\layersep,-\y cm) {};
    \node (O-3) at (2*\layersep,-3) {\vdots};

    \foreach \source in {-1,-2,-4}{
        \draw [black,-,dotted,thick,domain=-80:80] plot ({0.3+0.3*cos(\x)}, {\source+0.3*sin(\x)});
        \draw [black,-,dotted,thick,domain=-80:80] plot ({0.6+0.6*cos(\x)}, {\source+0.6*sin(\x)});
        \draw [black,-,dotted,thick,domain=-80:80] plot ({0.9+0.9*cos(\x)}, {\source+0.9*sin(\x)});
    }
   
    % Connect every node in the hidden layer with the output layer
    \foreach \source in {1,2,$K$}
        \foreach \dest in {1,2,$M$}
            \path (H-\source) edge (O-\dest);

\end{tikzpicture}
% End of code
\end{document}
