\documentclass[12pt]{standalone}
% Based on http://www.texample.net/tikz/examples/neural-network/

\usepackage{tikz}

\begin{document}

\def\layersep{2.5cm}

\begin{tikzpicture}[shorten >=1pt,->,draw=black!80, node distance=\layersep]
    \tikzstyle{every pin edge}=[<-,shorten <=1pt]
    \tikzstyle{neuron}=[circle,fill=black!25,minimum size=17pt,inner sep=0pt]
    \tikzstyle{phantomneuron}=[circle,fill=black!0,minimum size=17pt,inner sep=0pt]
    \tikzstyle{annot} = [text width=4em, text centered]

    % Draw the input layer nodes
    \foreach \name / \y in {1/1,2/2,$d$/4}
    % This is the same as writing \foreach \name / \y in {1/1,2/2,3/3,4/4}
        \node[phantomneuron] (I-\name) at (0,-\y) {\name};
    \node (I-3) at (0, -3) {\vdots};

    \node[neuron] (P) at (\layersep, -2 cm) {};
    
    \node[phantomneuron] (O) at (2*\layersep,-2 cm) {$f(\mathbf{x})$};
    % Connect every node in the input layer with every node in the
    % hidden layer.
    \foreach \source in {1,2,$d$}
            \path (I-\source) edge (P);
    \path (P) edge (O);
   
    % Connect every node in the hidden layer with the output layer
    %\foreach \source in {1,2,$K$}
    %    \foreach \dest in {1,2,$M$}
    %        \path (H-\source) edge (O-\dest);

\end{tikzpicture}
% End of code
\end{document}
